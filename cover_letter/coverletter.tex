%!TeX TS-program = lualatex
%!TeX encoding = UTF-8 Unicode
%!TeX spellcheck = fr-FR
%!BIB TS-program = biber
% -*- coding: UTF-8; -*-
% vim: set fenc=utf-8
%----------------------------------------------------------------------------------------
%	PACKAGES AND OTHER DOCUMENT CONFIGURATIONS
%----------------------------------------------------------------------------------------
\documentclass[10pt,english]{article}

\usepackage{charter,graphicx}
\usepackage[margin=1in]{geometry}

\usepackage{hyperref,tabularx}
\usepackage{fancyhdr}
\pagestyle{fancy}
\renewcommand{\headrulewidth}{.4pt}% Default header rule
\renewcommand{\footrulewidth}{0pt}% No footer rule
\fancyhf{}% Clear header/footer
\fancypagestyle{plain}{
  \renewcommand{\headrulewidth}{0pt}% No header rule
  \renewcommand{\footrulewidth}{.4pt}% Default footer rule
  \fancyhf{}% Clear header/footer
}
\AtBeginDocument{\thispagestyle{plain}}

\setlength{\parindent}{0pt}
\setlength{\parskip}{.5\baselineskip plus 1pt minus 1pt}
\usepackage[utf8]{inputenc}%
\usepackage{babel}%
\usepackage{etoolbox}
\makeatletter
\patchcmd{\@zfancyhead}{\fancy@reset}{\f@nch@reset}{}{}
\patchcmd{\@set@em@up}{\f@ncyolh}{\f@nch@olh}{}{}
\patchcmd{\@set@em@up}{\f@ncyolh}{\f@nch@olh}{}{}
\patchcmd{\@set@em@up}{\f@ncyorh}{\f@nch@orh}{}{}
\makeatother
%\usepackage{kpfonts}
%\usepackage{babel}
%\usepackage{csquotes}
\usepackage{url}
\usepackage{charter} % Use the Charter font for the document text
%----------------------------------------------------------------------------------------
%	YOUR NAME AND CONTACT INFORMATION
%----------------------------------------------------------------------------------------
%Jean-Nicolas Jérémie\\
%Institute of Neuroscience of la Timone\\
%CNRS - Aix Marseille University\\
%27, boulevard Jean Moulin\\
%13005 Marseille, France
\newcommand{\LastName}{Grimaldi}%{Perrinet}%
\newcommand{\FirstName}{Antoine}%{Laurent U.}%
\newcommand{\Institute}{Institut de Neurosciences de la Timone, CNRS / Aix-Marseille Universit\'e}%
\newcommand{\Address}{27, Bd. Jean Moulin, 13385 Marseille Cedex 5, France}%
\newcommand{\Website}{\url{https://laurentperrinet.github.io/}}%
\newcommand{\Email}{\url{laurent.perrinet@univ-amu.fr}}%
%----------------------------------------------------------------------------------------
%	YOUR DOCUMENT
%----------------------------------------------------------------------------------------
\usepackage[document]{ragged2e}
\begin{document}
%\centerline{
\includegraphics[width=.4\textwidth]{https://raw.githubusercontent.com/laurentperrinet/perrinet_curriculum-vitae.tex/master/troislogos.png}
%}
\\
\vspace{.1\baselineskip}
\hrulefill
\vspace{.1\baselineskip}

\begin{flushright}
	\FirstName\  \LastName\  \\
	\Institute\\[6pt]
	\Address\\%[6pt]
	\Website \\
	  Email: \Email \\[6pt]
\end{flushright}
\justifying
\vspace{1\baselineskip}
%----------------------------------------------------------------------------------------
%	ADDRESSEE AND GREETING/CLOSING
%----------------------------------------------------------------------------------------
%À qui de droit,\\
Marseille, 
November 30, 2022%\today
\\[12pt] % Date
	
Dear editor,

%----------------------------------------------------------------------------------------
%	LETTER CONTENT
%----------------------------------------------------------------------------------------
Please find enclosed a manuscript entitled ``Learning heterogeneous delays in a layer of spiking neurons for fast motion detection'' for your consideration as an article in \emph{Biological Cybernetics}. All the authors have been involved with the work, have approved the manuscript and agreed to its submission.

Our manuscript describes a novel method inspired by neuroscience to overcome some challenges faced in computer vision, notably when dealing with large amounts of data. By using heterogeneous delays on different synapses, this novel spiking neuron method is able to detect spiking motifs and we validate" the method on synthetic and then realistic data. Results show that this method could provide a path for future spiking neural network algorithms using less energy for a similar performance as their analog counterparts.

Our  manuscript follows our submission and presentation during the ``NeuroVision'' workshop. In addition, all the python code needed to reproduce figures and supplementary materials will be fully open-source

%\vspace{.2\baselineskip}
%----------------------------------------------------------------------------------------
Sincerely yours,%\vspace{.1cm}

\FirstName\ \LastName \\

%\includegraphics[width=.3\linewidth]{../RTC/signature_LuP.jpg}


\end{document}

