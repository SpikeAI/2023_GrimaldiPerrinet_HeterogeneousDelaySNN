%!TeX TS-program = lualatex
%!TeX encoding = UTF-8 Unicode
%!TeX spellcheck = fr-FR
%!BIB TS-program = biber
% -*- coding: UTF-8; -*-
% vim: set fenc=utf-8
%----------------------------------------------------------------------------------------
%	PACKAGES AND OTHER DOCUMENT CONFIGURATIONS
%----------------------------------------------------------------------------------------
\documentclass[10pt,english]{article}

\usepackage{charter,graphicx}
\usepackage[margin=1in]{geometry}

\usepackage{hyperref,tabularx}
\usepackage{fancyhdr}
\pagestyle{fancy}
\renewcommand{\headrulewidth}{.4pt}% Default header rule
\renewcommand{\footrulewidth}{0pt}% No footer rule
\fancyhf{}% Clear header/footer
\fancypagestyle{plain}{
  \renewcommand{\headrulewidth}{0pt}% No header rule
  \renewcommand{\footrulewidth}{.4pt}% Default footer rule
  \fancyhf{}% Clear header/footer
}
\usepackage%[disable]	% uncomment to hide all the notes
	{todonotes}
\newcommand{\noteRev}[1]{\todo[color=blue!10, inline]{#1}}

\AtBeginDocument{\thispagestyle{plain}}

\setlength{\parindent}{0pt}
\setlength{\parskip}{.5\baselineskip plus 1pt minus 1pt}
\usepackage[utf8]{inputenc}%
\usepackage{babel}%
\usepackage{etoolbox}
\makeatletter
\patchcmd{\@zfancyhead}{\fancy@reset}{\f@nch@reset}{}{}
\patchcmd{\@set@em@up}{\f@ncyolh}{\f@nch@olh}{}{}
\patchcmd{\@set@em@up}{\f@ncyolh}{\f@nch@olh}{}{}
\patchcmd{\@set@em@up}{\f@ncyorh}{\f@nch@orh}{}{}
\makeatother
%\usepackage{kpfonts}
%\usepackage{babel}
%\usepackage{csquotes}
\usepackage{url}
\usepackage{charter} % Use the Charter font for the document text
%----------------------------------------------------------------------------------------
%	YOUR NAME AND CONTACT INFORMATION
%----------------------------------------------------------------------------------------
%Jean-Nicolas Jérémie\\
%Institute of Neuroscience of la Timone\\
%CNRS - Aix Marseille University\\
%27, boulevard Jean Moulin\\
%13005 Marseille, France
\newcommand{\LastName}{Grimaldi}%{Perrinet}%
\newcommand{\FirstName}{Antoine}%{Laurent U.}%
\newcommand{\Institute}{Institut de Neurosciences de la Timone, CNRS / Aix-Marseille Universit\'e}%
\newcommand{\Address}{27, Bd. Jean Moulin, 13385 Marseille Cedex 5, France}%
\newcommand{\Website}{\url{https://laurentperrinet.github.io/}}%
\newcommand{\Email}{\url{antoine.grimaldi@univ-amu.fr}}%
%----------------------------------------------------------------------------------------
%	YOUR DOCUMENT
%----------------------------------------------------------------------------------------
\usepackage[document]{ragged2e}
\begin{document}
%\centerline{
\includegraphics[width=.4\textwidth]{~/Documents/troislogos.png}
%}
\\
\vspace{.1\baselineskip}
\hrulefill
\vspace{.1\baselineskip}

\begin{flushright}
	\FirstName\  \LastName\  \\
	\Institute\\[6pt]
	\Address\\%[6pt]
	\Website \\
	  Email: \Email \\[6pt]
\end{flushright}
\justifying
\vspace{1\baselineskip}
%----------------------------------------------------------------------------------------
%	ADDRESSEE AND GREETING/CLOSING
%----------------------------------------------------------------------------------------
%À qui de droit,\\
Marseille, 
February 10, 2023%\today
\\[12pt] % Date
	
Dear editor,

%----------------------------------------------------------------------------------------
%	LETTER CONTENT
%----------------------------------------------------------------------------------------
Please find enclosed the revised manuscript entitled ``Learning heterogeneous delays in a layer of spiking neurons for fast motion detection'' for your consideration as an article in \emph{Biological Cybernetics}. All the authors have been involved with the work, have approved the manuscript and agreed to its revision. The original submission ID is a7f45f8e-87d6-4a30-ac7f-9f0c94e2592a. 

Our manuscript describes a novel method inspired by neuroscience to overcome some challenges faced in computer vision, notably when dealing with large amounts of data. By using heterogeneous delays on different synapses, this novel spiking neuron method is able to detect spiking motifs and we validate the method on synthetic event-based data. Results show that this method could provide a path for future spiking neural network algorithms using less energy for a similar performance as their analog counterparts. This work is the extension of a previous study on a simpler event-based dataset. Our contribution here is to offer a better formal description of the model, an application on a more ecological task and testing of the influence of various parameters of the validation dataset.

Given the substantial modifications we did to the original manuscript to reduce and clarify the different sections we do not provide a tracked changes version of the revision (such a document can be provided if needed). We provide a point by point response to the reviewers with references to the manuscript.  

%\vspace{.2\baselineskip}
%----------------------------------------------------------------------------------------
Sincerely yours,%\vspace{.1cm}

\FirstName\ \LastName \\

\newpage

\textbf{REVIEWER REPORTS}

\textbf{Reviewer 1}

 1.There is a lack of an intuitive figure to show the HD-SNN model used in this paper.
 
 \noteRev{We provide such a figure in the new manuscript: Figure 3, P9}

 2.The MLR model used in the experimental part was presented in authors’ another work. Therefore, the changes on the MLR model are suggested to be highlighted and contrasted graphically.
 
 \noteRev{The main differences with the previous study are in the task to solve (section 2.1) and in the analysis of the results. The response of the model is tested on different sets of motion cloud stimuli with a control of the parameters of such stimuli generation (section 3.2). Regarding the MLR model, main differences with the previous ones are reported in the methods (P8, L269).}

 3.The title of section 2.2.1 is ’’a generative model for rater plots’’, but actually this section introduces basic terms and symbols. Section 2.2.2 is named ‘’detecting spiking motifs’’, but most of the content seems to introduce the generation process of raster plots, while only the last paragraph introduces the detection of motif. It is recommended that the content of these two sections be reorganized and that references to Figure 1(b)(c)(d) be added where appropriate during the description.
 
 \noteRev{We removed these sections to have a more concise explanation of the model.}

 4.The adjustment of weights and delays needs more detailed elaboration, preferably given in mathematical formulas.
 
 \noteRev{We improved the mathematical formulation of the model in sections 2.2.1 and 2.2.2 and hope that it clarifies the general framework we use. Regarding training, it is described more clearly in P8, L295. Demonstration of the mechanism and the analogy with Hebbian learning was done in~\cite{grimaldi_robust_2022} and was not reproduced here. We hope this new version offers a clearer methodology.}

 5.In Figure 4, it is claimed that the kernels are very similar for the ON polarities, and different kernels are selective to the different motion directions. However, in terms of these two observations, I think they are not very significant, so I suggest a more intuitive way to reflect them, such as quantifying the similarity between the ON kernels, and the angle between the kernels and the vector of the standard motion directions. 
 
 \noteRev{We gave a better explanation of the symmetry assumption in the methods (P8, L293) an how this new model uses this invariance in its computation like observed in complex cells. Indded, we don't want the polarity to have an impact on the classification and then this observation does not apply to the revised manuscript as it is a prior and not an emergent property.}

 6.There are few experiments and lack of comparative experiments. Since the model has an improvement based on MLR model, it is suggested to increase the comparison with MLR model and HOTS model on other datasets, such as N-MNIST and other neuromorphic datasets, so as to show the advantages of the proposed model in computational consumption.
 
 \noteRev{The reviewer is making a good point but regarding the deadline for the revision and some objectives we focused on to improve the manuscript we could not apply the model on DVS recordings.}

 7.Some sections are verbose, such as the discussion.
 
 \noteRev{We significantly changed the introduction (section 1), the methods (section 2) and the discussion (section 4) to address this comment. We hope to provide a more concise and clear explanation of our study.}

 8.There are some minor errors in the text, and it is recommended to check the full text: 
 1)In the first paragraph on page 8, "a presynaptic address $b_s$" should be "a postsynaptic address $b_s$"; 
 2)Whether $W_b$ in the first formula should be $K_b$? And d in it lacks description. 
 3)"PGs" should be given in its full name when it first appears. 
 4)There is no subgraph label of (a), (b) and (c) in Fig. 3, but the expression of Fig. 3-(a)/(b)/(c) appears many times. 
 5)The content of paragraph 1 on page 13 is repeated. 
 
 \noteRev{We corrected these minor errors and thank the reviewer for reporting them.}
 
 \noteRev{Overall, we thank Reviewer 1 for its comments that allowed us to focus on reshaping the manuscript and give clearer and shorter explanation of the work we submit in your journal. Because we did major modifications in the manuscript, we do not provide a tracked changes version of the revision but we can easily provide one if needed. }

\textbf{Reviewer 2}

 Key results: Please summarize what you consider to be the outstanding features of the work.
 Using a single layer of spiking neurons with heterogeneous delays to learn spatio-temporal  spiking motifs

 Validity: Does the manuscript have flaws which should prohibit its publication? If so, please provide details.

 The work presented here is very similar to the author's other work,  "Learning hetero-synaptic delays for motion detection in a single layer of spiking neurons" 10.1109/ICIP46576.2022.9897394. This must be addressed and cited with an explanation of how this work presented is an extension of this previous publication. Abstract, Fig 6 and the main result is presented in the previous work. Arguably some of the figures from the previous publication would be nice to see in this publication as it helps with the visualisation of the problem. One suggestion for the extension to make the work unique would be to extend with some event-camera data, as it is mentioned multiple times within the publication. 
 There are also a number of citations that could have been added that did appear in the other publication, while also including other motion detection/ segmentation or previous work on event-based optical flow, which is a very similar problem.

 Originality and significance: If the conclusions are not original, please provide relevant references.

 See above...

\noteRev{We thank Reviewer 2 for its comments and indeed this previous work was not mentioned in the original manuscript. As mentioned in the cover letter and in the previous responses, we add this citation and describe the main differences with the submitted study: ecological task (section 2.1), more complete formulation of the model (section 2.2) and list of the priors used in this particular model (P8, L269). We also cite studies that use similar methods (P7, L243). }

\end{document}

