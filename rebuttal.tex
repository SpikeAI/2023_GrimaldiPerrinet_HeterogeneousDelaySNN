%!TeX TS-program = Lualatex
%!TeX encoding = UTF-8 Unicode
%!TeX spellcheck = en-US
% https://tex.stackexchange.com/questions/354194/how-to-edit-newlfm 
\documentclass[9pt]{article}

\usepackage{charter,graphicx}
\usepackage[margin=1in]{geometry}

\usepackage{fancyhdr}
\pagestyle{fancy}

\newcommand{\AuthorAG}{Antoine Grimaldi}
\newcommand{\AuthorCB}{Camille Besnainou}
\newcommand{\AuthorHL}{Hugo Ladret}
\newcommand{\AuthorLP}{Laurent U Perrinet}
\newcommand{\EmailLP}{laurent.perrinet@univ-amu.fr}
\newcommand{\EmailAG}{antoine.grimaldi@univ-amu.fr}
\newcommand{\EmailHL}{hugo.ladret@univ-amu.fr}
\newcommand{\orcidLP}{0000-0002-9536-010X}
\newcommand{\orcidHL}{0000-0001-7999-3751}
\newcommand{\orcidAG}{0000-0002-3107-4788}
\newcommand{\Address}{Institut de Neurosciences de la Timone (UMR 7289); Aix Marseille Univ, CNRS; Marseille, France}%
\newcommand{\Affiliation}{Institut de Neurosciences de la Timone (UMR 7289); Aix Marseille Univ, CNRS;}%
\newcommand{\CityLP}{Marseille, France}%
\newcommand{\WebsiteLP}{https://laurentperrinet.github.io}%

\renewcommand{\headrulewidth}{.4pt}% Default header rule
\renewcommand{\footrulewidth}{0pt}% No footer rule
\fancyhf{}% Clear header/footer
\fancypagestyle{plain}{
  \renewcommand{\headrulewidth}{0pt}% No header rule
  \renewcommand{\footrulewidth}{.4pt}% Default footer rule
  \fancyhf{}% Clear header/footer
}
\AtBeginDocument{\thispagestyle{plain}}

\setlength{\parindent}{0pt}
\setlength{\parskip}{.5\baselineskip plus 1pt minus 1pt}
%: %%%%%%%%%%%%%%%%%%%%%%%%%%%%%%%%%%%%%%%%%%%%%%%%%%%%%%%%%%%%%%%%%%%%
%%%%%%%%%%%%%%%%%%%%%%%%%%%%%%%%%%%%%%%%%%
%%%%%%%%%%%%%%%%%%%%%%%%%%%%%%%%%%%%%%%%%%
\newcommand{\Journal}{IEEE International Conference on Image Processing}%

\bibliography{biblio}

\begin{document}

\hrulefill

\vspace{.1\baselineskip}

\begin{tabular}[b]{@{} l @{}}
  \today\\[12pt] % Date

\AuthorAG\ \\

\Address\\%[6pt]
%\AuthorA\\[6pt]
%%    \AddressA
%    \LongAddressA
%        \phonefrom{\PhoneA}
  Email: \EmailAG %\\[6pt]

\end{tabular}
%\hfill
%\begin{tabular}[t]{@{} l @{}}
%  Mrs. Jane Smith \\ % Addressee of the letter above the to address
%  Recruitment Officer \\ % To address
%  The Corporation \\
%  123 Pleasant Lane \\
%  City, State 12345
%\end{tabular}
%\hspace*{7em}

\hrulefill

\vspace{1\baselineskip}

To the Area Chair of \emph{\Journal},%

\vspace{2\baselineskip}

We would like to thank the reviewers for their comments that help us improve the clarity of our short paper. We are thankful to the reviewers for their global interest and the positive feedbacks raised by our study. %Even if some reviewers explicitly mention their lack of knowledge in the field preventing them to provide a confident evaluation, reviews are encouraging. 
We report in this rebuttal three main remarks brought by reviewers: 1/ about the pertinence of the contribution of this work to the image processing field (0357), 2/ about the clarity of the description of our model and about the figure illustrating it (1B91) and 3/ about the generalization of this model to more realistic data (1B91). 

First, we would like to highlight the fact that our work is to be presented at the Special Session about \textit{Neuromorphic and perception-based image acquisition and analysis}. Thus, using as a visual signal the output of an event-based camera fits the scope of this session. This is why we convert motion clouds movies into spikes. We are mostly ready to make some modifications to the manuscript to improve this description. We strongly believe such event-based algorithms will benefit the image processing community. 

Second, we will improve the formalism and the figure to clarify some confusing points. From the question, we understand there is some confusion about the scale of Figure 2-(a) and (c). Figure 2-(c) is the spiking output of the layer of hetero-synaptic delays neurons with the raster plot in Figure 2-(a) as input. Indeed, the output spike could not happen before the integration of the whole spatio-temporal motif. 

Third, the reviewer points to a really interesting question about generalization, and we would clearly like to go into this direction. Our future goal is to apply this model for more realistic data, and preliminary results are encouraging. As it stands, our model may not be able to infer a velocity that is different from the 8 motion directions learned during training. Indeed, we use a softmax activation function, which implies some lateral inhibition of the neurons of a layer. By changing the spiking mechanism, we believe that this model will be able to infer a linear combination of the basis motion directions learned to generalize to a continuum of motions. 

We hope these points clarify our contribution. Thank you for your consideration,
\vspace{\baselineskip}

Sincerely,

\vspace{2\baselineskip}

On behalf of the authors, \\
\AuthorAG

%%%%%%%%%%%%%%%%%%%%%%%%%%%%%%%%%%%%%%%%%%%%%%%%%%%%%%%%%%%%%%%%%%%%%
\vspace{.5\baselineskip}
\begingroup
\renewcommand{\section}[2]{}%
%\printbibliography
\endgroup
%%%%%%%%%%%%%%%%%%%%%%%%%%%%%%%%%%%%%%%%%%%%%%%%%%%%%%%%%%%%%%%%%%%%%
\end{document}
%%%%%%%%%%%%%%%%%%%%%%%%%%%%%%%%%%%%%%%%%%%%%%%%%%%%%%%%%%%%%%%%%%%%%
